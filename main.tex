\documentclass[12pt]{report}
\usepackage[utf8x]{inputenc}
\usepackage[serbianc]{babel}
\usepackage {graphicx}
\usepackage{float}
\usepackage{gensymb}
\def\figurename{slika}
\usepackage[section]{placeins}
\usepackage{makecell}
\usepackage{amsmath}
\usepackage[a4paper, total={6.5in, 9.4in}]{geometry}
\newtheorem{theorem}{\bf Теорема}[section]
\newtheorem{lemma}[theorem]{\bf Лема}
\newtheorem{corollary}[theorem]{\bf Последица}
\newtheorem{definition}[theorem]{\bf Дефиниција}
\newtheorem{conjecture}[theorem]{\bf Хипотеза}
\newtheorem{remark}[theorem]{\bf Примедба}
\newtheorem{problem}[theorem]{\bf Проблем}
\newtheorem{alg}[theorem]{\bf Алгоритам}
\newtheorem{pri}{\bf Пример}
\newtheorem{tvrdjenje}{\bf Тврђење}

\newcommand{\comment}[1]{\ $[\![${\normalsize #1}$]\!]$ \ }
\newcommand{\proof}{\noindent{\bf Доказ\ }}
\newcommand{\qed}{\hfill $\square$ \bigskip}
\def\cp{\,\square\,}



\usepackage[colorlinks = false,
            linkcolor = blue,
            urlcolor  = blue,
            citecolor = blue,
            anchorcolor = blue]{hyperref}


\begin{document}


\newcommand{\HRule}{\rule{\linewidth}{0.5mm}} % Defines a new command for the horizontal lines, change thickness here

\thispagestyle{empty}
\\
\centerline{\LARGE \textbf{МАТЕМАТИЧКА ГИМНАЗИЈА}}

\vspace*{40mm}
\begin{center}
{\Large МАТУРСКИ РАД}\\
{\Large	из предмета}\\
{\Large	aнализа са алгебром}\\
{\Large	на тему}
\Huge\HRule\\[0.4cm] %0.4cm
	{Генерисање поплочавања еуклидске равни}\\
	\HRule \\[20pt] %20pt
\begin{minipage}{0.4\textwidth}
\begin{flushleft} \large
\emph{\Large Ученик:}\\
{\Large Марина Васиљевић}, IV$_\text{Е}$\\
\end{flushleft}
\end{minipage}
~
\begin{minipage}{0.4\textwidth}
\begin{flushright} \large
\vspace{0.5cm}
\emph{\Large Ментор:} \\
{\Large Зоран Лучић}\\ 
\end{flushright}
\end{minipage}\\[8cm]
\Large{Београд, мај 2019.}
\end{center}

\tableofcontents

\thispagestyle{empty}

\clearpage
\setcounter{page}{1}

\chapter{Увод}
Људи су одувек имали потребу за украшавањем простора у којем бораве.  Откада је човек почео да користи камен за облагање зидова и подова, почео је и да бира камење по боји и облику и на тај начин прави шаре.
У свим земљама могу се наћи површи прекривене облицима који се слажу и стварају неку шару. Понављање неких облика доста је заступљено у орнаменталној уметности и мозаику.
Стари Грци и Римљани правили су мозаике који су приказивали сцене из историје, митологије или свакодневног живота. (слика 1). Са друге стране, Арапи и Маври користили су плочице у само неколико боја и облика и на тај начин правили занимљиве шаре којима су украшавали своје грађевине. Један такав пример је палата Алхамбра у Шпанији (слика 2). Данас се за поплочавања обично користе једноставни облици али се уметници као што је Морис Ешер \footnote{Maurits Cornelis Escher - холандски уметник и графичар, посебно познат по својим представама парадоксалних и немогућих призора (1898—1972)} баве занимљивим понављајућим шарама, као видом уметности (слика 3).
Иако сe дизајнирањем поплочавања људи баве хиљадама година, тек крајем 19. века настају први помаци у њиховом математичком објашњењу. (ВАЉДА?)


\chapter{Mатематикa поплочавања}\label{kristalografske-grupe-i-poploux10davanje}
\section{Математичко представљање поплочавања, рецимо}
Поплочавање је прекривање равни или простора геометријским фигурама чије се унутрашњости не преклапају. Такве фигуре називамо плочицама или теселима од латинске речи \emph{tessalla} која означава парче камена или стакла од кога се слаже мозаик. Посебно
су занимљива поплочавања код којих су плочице подударне, односно за сваке две плочице постоји пресликавање које чува растојања између тачака, односно \emph{изометријска трансформација}. Такво поплочавање назива се \emph{моноедарско}.

У овом раду ћемо се бавити поплочавањима еуклидске равни
\(\mathbb{E}^2\). Прво ћемо дефинисати појам изометријске трансформације.

\begin{definition}
Нека је $\sigma \! : \, \mathbb{E}^2 \xrightarrow \mathbb{E}^2$ је изометријска трансформација и $A$ и $B$ произвољне тачке равни, тада је растојање тачака $A$ и $B$ једнако растојању  $\sigma (A)$ и $\sigma(B)$.
\end{definition}

Све изометрије равни су једног од четири типа:
\begin{enumerate}
    \item Транслација за дати вектор, односно за дати смер и растојање (слика)
    \item Ротација око тачке $O$, која се назива \emph{центар ротације} за дати угао $\varphi$. Специјално, када је $\varphi = \pi$ назива се \emph{централна симетрија} у тачки $O$.
    \item Рефлексија са осом $p$, назива се и \emph{осна симетрија} (слика)
    \item Клизајућа рефлексија, односно композиција осне симетрије и транслације дуж осе те симетрије за одређено растојање (слика).
\end{enumerate} 

Изометрија је \emph{директна} ако задржава оријентацију, односно ако се сваки троугао равни слика у троугао који је исто оријентисан (слика). Директне изометрије су транслација и ротација. Рефлексија и клизајућа рефлексија су \emph{индиректне} што значи да се сваки троугао слика у троугао супротне оријентације (слика

%Изометрија која сваку тачку слика у саму себе назива се \emph{коинциденција}.

Како се композицијом изометријских трансформација добија изометријска трансформација, скуп свих изометрија са операцијом композиције је алгебарска структура група. Сваку подргрупу те групе називамо групом изометрија. 

Симетрија неке фигуре $Т$ је изометрија $\sigma$ таква да слика $Т$ у саму себе односно $\sigma Т = Т$. Како је композиција симетрија такође симетрија, скуп свих симетрија једне фигуре је група. На примеру видимо симетрије нечега.... **КАД БУДЕМ ДОДАВАЛА СЛИКЕ**

Симетрија поплочавања $\tau$ је изометрија која сваку плочицу тог пресликавања пресликава на неку другу плочицу.*стићи ће пример некад*
 ???
Посматрајмо поплочавање које има две симетрије које су неколинеарни вектори $a$ и $b$. Тада група поплочавања садржи и све векторе $na + mb$. Применом тих изометрија на произвољну тачку добијам решетку са паралелним ивицама односно поплочавање паралелограмима (слика).  Један такав паралелограм назива се \emph{периодични паралелограм} па је и свако овакво поплочавање \emph{периодично}. Уколико нам је познат периодични паралелограм за неко поплочавање можемо поплочати целу раван користећи само транслације за векторе $na + mb$. Добијене решетке подсећају на кристалне решетке па се такве групе називају \emph{кристалографске групе}.

Постоје и поплочавања која нису периодична, односно које немају симетрије које су неколинеарни вектори. На сликама се виде примери неких апериодичних поплочавања али се њима у даљњм раду нећемо бавити и под поплочавањем ћемо подразумевати периодично поплочавање. (слика)

Групу симетрија можемо представити преко дијаграма, приказујући њене изометрије користећи табелу**
За два дијаграма сматрамо да су исти уколико се разликују само по транслацијама и оне се могу свести једне на друге \footnote{Пар вектора $a,b$ се може свести на $a',b'$ уколико је $a'=na, b'=mb$, где су $n$ и $m$ цели бројеви. Место вектора транслације на дијаграму није битно.}.(слика)
Две групе симетрија си изоморфне уколико се њихови дијаграми афиним трансформацијама могу свести један на други. 
Познато је да над \(\mathbb{E}^2\) постоји 17 класа изоморфности кристалографских група. 




%група симетрија фигуре
%група симетрија поплочавања
%периодична попличавања
%фундаментална област групе
%кристалографкса група-кончнп генерисана и постоји полигална ф област


%тако да постоји група изометрија које једну фигуру пресликавају на све остале. Један од примера таквог поплочавања су кристалне решетке, због чега се такве групе називају кристалографским групама. 



%У оквиру овог рада бавићемо се својствима кристалографских група над \(\mathbb{E}^2\), својствима фундаменталних области и алгоритмима за конструисање фундаменталних полигона који на крају могу имати практичну примену у графичком дизајнирању поплочавања. Алгоритме ћемо имплементирати у програмском језику Пyтхон.

    \section{\texorpdfstring{Представљање изометријских трансформација 
\(\mathbb{E}^2\)}{Izometrijske transformacije u \textbackslash{}mathbb\{Е\}\^{}2}}\label{izometrijske-transformacije-u-mathbbr2}


За групу кажемо да је коначно генерисана ако постоји коначан подскуп елемената групе који генерише све елементе групе. У случају кристалографских група то је коначан скуп изометрија чијом композицијом добијамо све изометрије равни потребне за њено поплочавање датим фундаменталним доменом.

\begin{definition} Кристалографска група је коначно
генерисана периодична група изометрија.
\end{definition}


    Ако у \(\mathbb{E}^2\) уведемо координатни систем, тачке представљамо
дводимензионим координатама тј. елементима \(\mathbb{R}^2\). Координате
тацке \(A\) означаваћемо са \((A_x, A_y)\). Аналитички ћемо изометријску
трансформацију \(\mathbb{R}^2\) представити као специјалан случај
афине трансформације. Ако је \(A' = G(A)\) тада имамо:
\[A'_x = pA_x + qA_y + c_x\] \[A'_y = rA_x + sA_y + c_y\] Коришћењем
рачуна са матрицама то можемо представити:

\[\begin{bmatrix}p & q\\ r & s\end{bmatrix} \begin{bmatrix}A_x\\ A_y \end{bmatrix} + \begin{bmatrix}c_x\\ c_y\end{bmatrix} = \begin{bmatrix}A'_x\\ A'_y \end{bmatrix}\]

Коришћењем хомогених координата ово можемо представити као множење
проширених тродимензионих:

\[\begin{bmatrix}p & q & c_x\\ r & s&c_y \\ 0 & 0 & 1\end{bmatrix} \begin{bmatrix}A_x\\ A_y\\1\end{bmatrix} = 
\begin{bmatrix}A'_x\\ A'_y\\1\end{bmatrix}\]

На пример, транслација за вектор \((t_x, t_y)\) се представља као:

\[\begin{bmatrix}1 & 0 & t_x\\ 0 & 1&t_y \\ 0 & 0 & 1\end{bmatrix} \begin{bmatrix}A_x\\ A_y\\1\end{bmatrix} = 
\begin{bmatrix}A_x+t_x\\ A_y+t_y\\1\end{bmatrix}\]

Ротација око координатног почетка за угао $\varphi$ је

\[\begin{bmatrix}cos(\varphi) & sin(\varphi) & 0\\ -sin(\varphi) & cos(\varphi)&0 \\ 0 & 0 & 1\end{bmatrix} \begin{bmatrix}A_x\\ A_y\\1\end{bmatrix} = 
\begin{bmatrix}A_x  cos(\varphi) +  A_y  sin(\varphi)\\ -A_x  sin(\varphi) + A_y  cos(\varphi)\\1\end{bmatrix}\]

Рефлексију око $x$-осе представља матрица:
\[\begin{bmatrix}1 & 0 & 0\\ 0 & -1&0 \\ 0 & 0 & 1\end{bmatrix} \begin{bmatrix}A_x\\ A_y\\1\end{bmatrix} = 
\begin{bmatrix}A_x \\ -A_y\\1\end{bmatrix}\]

Предност  коришћцења  хомогених  координата  је  у  томе  што  слагање  изометрија  рачунамо  као множење матрица. Како ротацију око дате тачке $C$ можемо дефинисати као
$$R_{C,\varphi} = \tau_{\vec{OC}} \cdot R_{O,\varphi} \cdot \tau_{\vec{CO}}$$
њену хомогену матрицу рачунамо

\[\begin{bmatrix}1 & 0 & C_x\\ 0 & 1&C_y \\ 0 & 0 & 1\end{bmatrix}
\begin{bmatrix}cos(\varphi) & sin(\varphi) & 0\\ -sin(\varphi) & cos(\varphi)&0 \\ 0 & 0 & 1\end{bmatrix}
\begin{bmatrix}1 & 0 & -C_x\\ 0 & 1&-C_y \\ 0 & 0 & 1\end{bmatrix}
= 
\begin{bmatrix}cos(\varphi) & sin(\varphi) & -C_x cos(\varphi) - C_y sin(\varphi) + C_x\\ -sin(\varphi) & cos(\varphi)&C_x sin(\varphi) - C_y cos(\varphi) +y\\ 0 & 0 & 1\end{bmatrix}\]



\chapter{Конструкција фундаменталне области}\label{dirihleova-fundamentalna-oblast} 

До сада смо дефинисали групу за дато поплочавање али да бисмо могли да конструишемо, морамо представити поплочавање преко дате кристалографске групе. Такође, морамо имати плочицу над којом вршимо изометрије. Таква плочица назива се \emph{фундаментални домен}

\begin{definition} Коначна фундаментални домен \(F\) групе изометрија \(G\) еуклидске равни \(\mathbb{E}^2\) је унија коначно много полигона таквих да:\\
\begin{enumerate}
\item \(\displaystyle{\bigcup_{g\in G}g(F)} = \mathbb{E}^2\) 
\item \((\forall g\in G)(\mathring{F} \cap g(\mathring{F})= \emptyset)\) .
\end{enumerate}

\end{definition}
Фундаментална областа по дефиницији не мора да буде
полигон, али је за потребе овог рада довољно да се ограничимо на
полигоне. Такође, општија дефиниција захтева појмове из топологије,
док смо се овако задржали да геометријској терминологији.
Када је дата кристалографска група поставља се питање како можемо конструисати фундаменталне полигоне за ту групу. У овом поглављу ћемо описати Дирихлеов домен као и уопштења Дирихлеовог домена.
\section{Воронојев дијаграм}


Посматрајмо раван $\pi$ и две произвољне тачке $A$ и $B$ у њој. Поделимо је симетралом дужи $AB$ на две полуравни $\pi _A$ и $\pi _B$, тако да $A \in \pi _ A$ и $B \in \pi _B$. Јасно је да су све тачке на симетрали једнако удаљене од тачака $A$ и $B$, као и да је свакој тачки у $\pi _A$ ближа тачка $A$ него тачка $B$ и обрнуто. 
На тај начин смо све тачке равни $\pi$ поделили на два подскупа према томе којој од тачака $A$ и $B$ су ближе, а на рубу тих скупова (симетрала дужи $AB$) су тачке које су једнако удаљене од $A$ и $B$.

Уопштимо овај поступак за коначан број почетних тачака.

Означимо редом са $\pi _{XY}$ и $\pi _{YX}$ отворене полуравни којима симетрала дужи $XY$ дели раван $\pi$ на две полуравни тако да $X \in \pi _ {XY}$ и $Y \in \pi _{YX}$. 

Уколико на почетку имамо три тачке $A$, $B$ и $C$, тада је $\pi _{AB}$ скуп свих тачака у равни којима је тачка $A$ ближа него тачка $B$ и $\pi _{AC}$ скуп тачака којима је тачка $A$ ближа него тачка $C$, па је $\pi _{AB} \cap \pi _{AC}$ скуп тачака којима је тачка $A$ најближа тачка.

Применом овог поступка на  $n$ тачака, $A_1$, $A_2$, ... $A_n$ добијамо да је скуп тачака којима је $A_i$ најближа   $$\bigcap _{1\leq ј\leq н,\; ј\neq i} \pi_{A_i А_ј}.$$

Оваквку поделу равни на области према растојњу од датих тачака назива се \emph{Воронојевим дијаграмом} према руском матетичару \emph{Георгију Вороноју}.

\begin{definition}%(voronijev dijagram)
За скуп тачака равни $S$, Воронојев дијаграм је подела равни на затворене дисјунктне области $V_S(A_i)$, које називамо Воронојевим  ћелијама, таква да
$$ (\forall X \in V_{S}(A))(\forall B \in S\setminus \{A\})\quad d(X,A)\leq d(X,B) $$
$$ \bigcup_{\forall A \in S} V_{S}(A) = \mathbb{E}^2 .$$

\end{definition}

Из претходног разматрања произилази
$$V_S(A) = \bigcap _{B \in S \setminus \{A\}} \pi_{AB}.$$ 

\section{Дирихлеов домен}

Воронојеви дијаграми су један метод конструисања фундаменталног домена кристалографске групе $G$ тако што је $S$ орбита неке тачке $X$. Тада је Воронојев дијаграм поплочавање равни под дејством $G$, а свака ћелија дијаграма је фундаментални домен. \\
Такав фундаментални домен назива се \emph{Дирихлеов домен}.

Са $O_G(X)$ означавамо орбиту тачке $X$, односно
$$O_G(X) = \{g(X)\:|\:g \in G\} .$$

\begin{definition}
За дату тачку $X$ и кристалографксу групу $G$ Дирихлеов домен означавамо са $D_G(X)$ и важи
$$D_G(X) = \{Y \in \mathbb{E}^2\:|\:(\:\forall g \in G \setminus \{I\})\:(d(Y,X)\leq d(Y,g(X))\:\}.$$
\end{definition}

\noindent  Докажимо да овако дефинисан Дирихлеов домен јесте фундаментални домен за кристалографксу групу $G$.
Дирихлеов домен можемо посматрати као ћелију Воронојевог дијаграма 
$$D_G(X)= V_{O_G(X)}(X).$$

\noindent Како важи $d(g(X), g(Y))= d(X,Y)$ то је  $D_G(g(X)) = g(D_G(X))$  .
Такође приметимо да $$Y \in  O_G(X) \implies O_G(Y) = O_G(X) $$
На основу претходног важи

$$\bigcup_{g\in G}g(D_G(X)) = \bigcup_{g\in G}D_G(g(X)) = \bigcup_{y \in O_G(X)}D_G(Y)
= \bigcup_{Y \in O_G(X)}V_{O_G(Y)}(Y) = \mathbb{E}^2.$$

\noindent Тиме смо доказали први услов из дефиниције фундаменталног домена, да његове слике прекривају целу раван.

\noindent Слично закључујемо да за $Y = g(X)$
$$g(D_G(X)) = D_G(Y) = V_{O_G(Y)}(Y)$$ што значи да су $D_G(X)$ i $D_G(Y)$ различите ћелије Воронојевог дијаграма, па им сед унутрашњости не секу. 

Тиме смо доказали и други услов из дефиниције фундаменталног домена, да се његове слике не преклапају. Дакле, доказали смо да је $D_G(X)$ фундаментални домен групе $G$.

На наредним примерима приказани су примери Дирихлеових домена за групу \textbf{p6} и произвољне тачке. 
Група \textbf{p6} је генерисана ротацијом за угао \(\frac{2\pi}{3}\) око једног
и \(\frac{\pi}{3}\) око друга два темена троугла са угловима \(\frac{2\pi}{3}\), \(\frac{\pi}{6}\) i \(\frac{\pi}{6}\).



  \begin{figure}[H]
  \begin{subfigure}[b]{0.3\textwidth}
    \includegraphics[width=\textwidth]{output_10_0.png}
    \label{fig:f1}
  \end{subfigure}
  \begin{subfigure}[b]{0.3\textwidth}
    \includegraphics[width=\textwidth]{output_13_0.png}
    \label{fig:f2}
  \end{subfigure}
  \begin{subfigure}[b]{0.3\textwidth}
    \includegraphics[width=\textwidth]{output_12_0.png}
    \label{fig:f3}
  \end{subfigure}
\end{figure}


    \section{Уопштени Дирихлеов домен за више тачака}\label{konstrukcija-dirihleove-fundamentalne-oblasti}
    
Слично Дирихлеовом домену за једну тачку,  можемо посматрати домен генерисан двема тачкама. Тада нам је  $O_G(\{X,Y\})$ скуп тачака које се могу добити изометријама од једне од те две тачке. Посматрајмо Воронојев дијаграм над свим тачкама орбите $O_G(\{X,Y\})$.  Унија Воронојевих ћелија за тачку $X$ и за тачку $Y$ је фундаментални домен групе $G$, што је у општем случају исказано следећим тврђењем
\begin{tvrdjenje}
Нека је $S$ коначан скуп тачака равни на коју делује кристалографска група $G$, тада је 
$$D_G(S) = \{x \in \mathbb{E}^2\:|\:(\:\forall g \in G \setminus \{I\})\:(d(X,S)\leq d(X,g(S))\:\}$$

фундаментални домен групе $G$, где је $d(X,S) = \min_{Y \in S} d(X,Y)$ i va\v zi 
$$D_G(S) = \bigcup_{X \in S} V_{O_G(S)}(X) $$ 
\end{tvrdjenje}

%Doka\v zimo prvo drugi deo tvr\dj enja. 
%Posmatrajmu ta\v cku $Y$ takvu da $Y \in D_G(X)$. Tada je $(\:\forall X_1 \in O_G(S))\:(d(Y,X)\leq d(Y,X_1)$
%$$D_G(S) = \bigcup_{X \in S} D_G(X) $$

%nastavice se

\begin{samepage}
На примерима видимо уопштен Дирихлеов домен за две, три и четири произвољне тачке и групу \textbf{p6}.

\begin{figure}[H]
  \begin{subfigure}[b]{0.3\textwidth}
    \includegraphics[width=\textwidth]{output_14_0.png}
    \label{fig:f4}
  \end{subfigure}
  \begin{subfigure}[b]{0.3\textwidth}
    \includegraphics[width=\textwidth]{output_15_0.png}
    \label{fig:f5}
  \end{subfigure}
  \begin{subfigure}[b]{0.3\textwidth}
    \includegraphics[width=\textwidth]{output_16_0.png}
    \label{fig:f6}
  \end{subfigure}
\end{figure}
\end{samepage}

    \section{Уопштени Дирихлеов домен за полигон}\label{modifikacija-fundamentalne-oblasti-na-osnovu-podfundamentalne}
Дефиниција Дирихлеовог домена се може уопштити и када скуп $S$ није коначан, а конструкција Воронојевог дијаграма у том случају се може извести као гранична вредност када бирамо све већи број тачака.

Нека је дат полигон $P$ тако да се слике полигона у орбити не пресецају. 


Нека је $S_0$ скуп темена полигона $P$, а $S_n$ скуп у коме су поред темена полигона и по $n$ тачака са сваке од страница полигона на једнаким растојањима у оквиру странице.

На слици 7 је дат пример за $D_G(S_0)$.Он јесте фундаментални домен али приметимо да не обухвата цео полигон $P$.

На слици 8 је приказан $D_G(S_1)$, где су укључена и средишта страница и приметићујемо да тај фундаментални домен боље покрива почетни полигон. 


Уопштени Дирихлеов домен за полигон можемо да дефинишемо исто као и за коначан скуп
$$D_G(P) = \{X \in \mathbb{E}^2\:|\:(\:\forall g \in G \setminus \{I\})\:(d(X,P)\leq d(X,g(P))\:\}$$
при чему је $d(X,P) = \inf_{Y \in P} d(X,Y)$ i tada va\v zi
$$ D_G(P) = \lim _{n\to \infty} D_G(S_n). $$


За практичне потребе треба узети довољно велико $n$. На слици 9 дат је пример за $D_G(S_{50})$

  \begin{figure}[H]
  \begin{subfigure}[b]{0.3\textwidth}
    \includegraphics[width=\textwidth]{output_17_0.png}
    \label{fig:f7}
  \end{subfigure}
  \begin{subfigure}[b]{0.3\textwidth}
    \includegraphics[width=\textwidth]{output_18_0.png}
    \label{fig:f8}
  \end{subfigure}
  \begin{subfigure}[b]{0.3\textwidth}
    \includegraphics[width=\textwidth]{output_21_2.png}
    \label{fig:f9}
  \end{subfigure}
\end{figure}

\begin{samepage}
 На следећим сликама су дати примери уопштених дирихлеових домена за исти полигон у разним кристалографским групама.
 \begin{figure}[H]

  \begin{subfigure}[b]{0.3\textwidth}
    \includegraphics[width=\textwidth]{output_21_1.png}
    \label{fig:f20}
    \caption{grupa \textbf{p1}}
  \end{subfigure}
  \begin{subfigure}[b]{0.3\textwidth}
    \includegraphics[width=\textwidth]{output_21_2.png}
    \label{fig:f21}
    \caption{grupa \textbf{p2}}
  \end{subfigure}
  \begin{subfigure}[b]{0.3\textwidth}
    \includegraphics[width=\textwidth]{output_21_10.png}
    \label{fig:f24}
    \caption{grupa \textbf{pg}}
  \end{subfigure}

  \begin{subfigure}[b]{0.3\textwidth}
    \includegraphics[width=\textwidth]{output_21_4.png}
    \label{fig:f23}
    \caption{grupa \textbf{p4}}
  \end{subfigure}
  \begin{subfigure}[b]{0.3\textwidth}
    \includegraphics[width=\textwidth]{output_21_3.png}
    \label{fig:f22}
    \caption{grupa \textbf{p3}}
  
  \end{subfigure}
  \begin{subfigure}[b]{0.3\textwidth}
    \includegraphics[width=\textwidth]{output_21_7.png}
    \label{fig:f25}
    \caption{grupa \textbf{cmm}}
  \end{subfigure}
\end{figure}
\end{samepage}

\quad \\ \qquad
    \chapter{Имплементација}\label{implementacija}

  На основу претходно описаних метода конструкције фундаменталних домена имплементирана је рачунарска имплементација која омогућава интерактивну конструктцију фундаменталног домена за изабрану кристалографску групу.
Апликација се може користити за дизајнирање занимљивих поплочавања.
Имплементирана је у програмском језику Пајтон.

На приказаним изгледима екрана корисник додаваје тачку по тачку путем клика на мишу и на тај начин обликовује фундаменталну област. За конструкцију фундаменталних домена поред изабраних тачака коришћене су додатне тачке између њих како би облик фундаменталног домена био "мекши". 



\begin{figure}[H]

  \begin{subfigure}[b]{0.33\textwidth}
    \includegraphics[width=0.9\textwidth]{sl1.png}

  \end{subfigure}
  \begin{subfigure}[b]{0.33\textwidth}
    \includegraphics[width=0.9\textwidth]{sl2.png}

  \end{subfigure}
  \begin{subfigure}[b]{0.33\textwidth}
    \includegraphics[width=0.9\textwidth]{sl6.png}

  \end{subfigure}

\quad

  \begin{subfigure}[b]{0.33\textwidth}
    \includegraphics[width=.9\textwidth]{sl3.png}

  \end{subfigure}
  \begin{subfigure}[b]{0.33\textwidth}
    \includegraphics[width=.9\textwidth]{sl4.png}

  
  \end{subfigure}
  \begin{subfigure}[b]{0.33\textwidth}
    \includegraphics[width=.9\textwidth]{sl5.png}

  \end{subfigure}
  
\end{figure}

\chapter {Закључак}
У раду се бавимо конструисањем фундаменталних области кристалографских група еуклидске равни. Математички је описана метода конструкције заснована на уопштењу Дирихлеовог домена и имплементирана је рачунарска апликација која ефективно конструише фундаменталне домене коришћењем те методе. Корисник апликације кликом миша уцртава тачке. На основу задатих тачака програм констурише фундаментални домен коме припрадају све тачке равни које су ближе некој од задатих тачака него некој од њихових слика у изометријама из кристалографске групе и у реалном времену се исцртава добијено поплочавање. Апликација има опцију да изабрани скуп тачака посматра као темена изломљене линије, на основу које се конструише фундаментални домен као скуп тачака равни којима је та изломљена линија ближа неко било која њена слика. Тако добијен фундаментални домен је глаткији. 




\renewcommand\bibname{Литература}
\begin{thebibliography}{}

\bibitem{1}  Carne, T. K.  "Geometry and groups." Lecture notes, Cambridge University (2006). 
\bibitem{6}
Grünbaum B, Shephard GC."Tilings and patterns." Freeman (1987).
 \bibitem{4}  Kaplan, Craig S. "Introductory tiling theory for computer graphics."{} Synthesis Lectures on Computer Graphics and Animation 4.1 (2009): 1-113. 
 \bibitem{6}  Kilgore, J. "Fundamental Domains of Discrete Groups Acting on Euclidean Space." (2012). 
 \bibitem{2} Lučić, Z., and Molnár, E. "{}Combinatorial classification of fundamental domains of finite area for planar discontinuous isometry groups." Archiv der Mathematik 54.5 (1990): 511-520.
\bibitem{1} Molnár, E. "Nice tiling, nice geometry." Teaching Mathematics and Computer Science, Debrecen (2012): 269-280.


\bibitem{5} 
  Schattschneider, D. "The plane symmetry groups: their recognition
  and notation." \emph{The American Mathematical Monthly} 85.6 (1978):
  439-450.



\end{thebibliography}





	

\end{document}
